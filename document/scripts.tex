\chapter{Python Scripts and Functions}
\label{ch:scripts}

Make a directory where you will keep all of your scripts.  You will
also need to choose a text editor, such as Emacs, TextWrangler, NEdit,
or others.  You will be using this to edit your scripts.
 
\section{MyFirstScript.py}

Create a file called MyFirstScript.py in your script directory, and
open it in your favorite editor.  Comments in scripts are very
important, not just to help you remember what exactly you were
thinking, but to help others as well should they ever use your code.
Therefore, to start off on the right foot, let's create a header that
looks something like this:

\begin{verbatim}
' ' '
ABOUT:
This is a program that takes this and plots that.

DEPENDS:
Python 2.5.4

AUTHOR:
M.E. MySelf for STScI, 2011

HISTORY:
2011: Trial program.

USE:
python MyFirstScript.py
' ' '
\end{verbatim}

For the header we show three apostrophes to comment out multiple lines
of text.  Next thing we will want to have is all of our imports.

\begin{verbatim}
import numpy as np
import pylab as pl 
\end{verbatim}

Now we are ready!  Look back through your shell and create your script
from Chapter~\ref{ch:pylab}.  So far you should have something similar
to the following:

\begin{verbatim}
infile = 'Gordon2005_Fig16.txt'
outfile = 'fig16_log.pdf' 

slope, ran_slope_unc, corr_slope_unc, both_slope_unc,  \
    eqn_slope_unc = np.loadtxt(infile, 
    usecols=(0, 1, 2, 3, 4), unpack=True) 
pl.loglog(slope, ran_slope_unc, 'b--',  
    slope, corr_slope_unc,'r:' , slope, both_slope_unc, 'g-',  
    slope, eqn_slope_unc, 'm-.', linewidth=3) 
pl.legend(('Random Uncertainty','Correlated Uncertainty',  
    'Both','Equation'),'best')
pl.ylabel('Slope Unc. [e-/s]') 
pl.xlabel('Slope [e-/s]') 
pl.savefig(outfile) 
pl.clf()
\end{verbatim}

Notice that we listed any variables (or things we might want to change
later) at the top of the program, such as $outfile$ and $infile$.
Also, since we do not have to type the variables over and over, I
decided to use the more descriptive names.  This will also help if I
do not work on this script for awhile and then later come back to it.
Now our program is ready to execute.

\subsection{Executing Python Scripts}

There are a few ways to run a Python program.  One is to type:

\texttt{\termtab python MyFirstScript.py}

Or, if you are already inside the Python interactive environment, just
type:

\texttt{\pytab import MyFirstScript}

For the last example, if you want to re-run the script, type:

\texttt{\pytab reload(MyFirstScript)}

Another way is to make the program executable and then type:

\texttt{\termtab MyFirstScript.py}

This is nice.  You do not have to type `python', you can run it from
anywhere, and you do not have to be in the Python interactive
environment.  However, we must first do two things.
\begin{enumerate}
\item To tell your computer which shell or interpreter should be used
  for executing this file, at the very top of your file add:

  \texttt{\#! /usr/bin/env python}

\item You will need to make your script executable.  In the terminal,
  type:

  \texttt{\termtab chmod a+x MyFirstScript.py}  
\end{enumerate}

To be able to call your script from anywhere on your computer, add
your script directory to your executable path by opening your
.mysetenv file and adding the line (with the correct path substituted
in):

\texttt{setenv PATH .:/my/script/directory:\{\$PATH\}}

Next, in your terminal execute the command:

\texttt{\termtab source .mysetenv}

{\color{blue} {\sf\small EXERCISES}} \\
{\it Exercise \arabic{exercise} \stepcounter{exercise}:  \\
  Write your script MyFirstScript.py and execute it.
}

\section{Adding Functions}
\label{s:fun}
MyFirstScript.py is short and easy to read.  However, we can imagine a
case where we have such a large program and many repeated tasks that
it will become difficult.  For example, if we want to create the
y-intercept uncertainty plot as well as the slope uncertainty plot we
can either repeat several lines of our program (and if we change one
copy remember to change the other), or we can create a plotting
function and just call that function twice.  The general format of
declaring a function and then using it is shown below.

\begin{verbatim}
#! /usr/bin/env python

# Header

def mkplot(data): 
    # Make plot here
    return

if __name__=='__main__':
    data = SomeFunctionThatGetsData()
    mkplot(data)
    print 'Now I have a beautiful plot!'
\end{verbatim}

Notice the line \texttt{if \_\_name\_\_=='\_\_main\_\_':}.  It is often useful
to start the main program with this statement.  In this way you can
make the file usable as a script as well as an import-able module,
because the code that parses the command line only runs if the module
is executed as the `main' file, i.e. `MyFirstScript.py'.  In this way,
from another program (or a Python prompt) we could call

\texttt{from MyFirstScript import mkplot}.   

Let's modify our program to look like this:

\begin{verbatim}
#! /usr/bin/env python

#Header

__author__ = 'M.E. MySelf'
__version__ = 0.2

import numpy as np
import pylab as pl

def mkplot(outfile,xx,yy1,yy2,yy3,yy4,ylab='Slope Unc. [e-/s]'):
    pl.loglog(xx, yy1, 'b--', xx, yy2, 'r:' , xx, yy3, 'g-', xx,  
        yy4, 'm-.', linewidth=3) 
    pl.legend(('Random Uncertainty','Correlated Uncertainty',  
        'Both','Equation'),'best')
    pl.ylabel(ylab) 
    pl.xlabel('Slope [e-/s]') 
    pl.savefig(outfile) 
    pl.clf()
    print 'Saved file to: ',outfile 
    return  
    
if __name__=='__main__': 
    infile = 'Gordon2005_Fig16.txt' 
    slope_outfile = 'fig16_slope.pdf'  
    yint_outfile = 'fig16_yint.pdf'  
    
    slope, ran_slope_unc, corr_slope_unc, both_slope_unc,  \
        eqn_slope_unc, ran_yint_unc, corr_yint_unc,  \
        both_yint_unc, eqn_yint_unc = np.loadtxt(infile, 
        unpack=True) 
    mkplot(slope_outfile, slope, ran_slope_unc, corr_slope_unc,  
        both_slope_unc, eqn_slope_unc) 
    mkplot(yint_outfile, slope, ran_yint_unc, corr_yint_unc,  
        both_yint_unc, eqn_yint_unc,
        ylab='Y-Intercept Unc. [e-]')
\end{verbatim}

I also added \texttt{\_\_author\_\_ = 'M.E. MySelf'} and
\texttt{\_\_version\_\_ = 0.2}.  These lines are not necessary, but
they mean I can do the following:
\begin{alltt}
\pytab import MyFirstScript
\pytab MyFirstScript.__version__
\pytab MyFirstScript.__author__
\end{alltt}

{\color{blue} {\sf\small EXERCISES}} \\
{\it Exercise \arabic{exercise} \stepcounter{exercise}:  \\
  What version of NumPy are you using?  Does NumPy list an author?
}

Notice in the definition of our mkplot function that we give the
variable `ylab' a default value.  Now we only have to assign a y-axis
label when we do not want the default (i.e. the y-intercept case).
Finally, {\sf \small mkplot} is a function, but not a module.

\section{Passing Arguments on the Command Line and {\sf argparse}}
What if we wanted to run this program, but for other input files?  A
simple solution would be to just open up our script and edit the file
name.  However, a more user friendly method would be to allow it to be
entered on the command line.  {\sf\small argparse} is a module that
makes this process easy.  Look at the use of {\sf\small
  argparse} in the code below.  This code can be run in the following
ways:
\begin{alltt}
\termtab MyFirstScript.py 
\termtab MyFirstScript.py --help
\termtab MyFirstScript.py -f Gordon2005_Fig16.txt
\termtab MyFirstScript.py --file Gordon2005_Fig16.txt
\end{alltt}
For more information, check out the link
\url{http://docs.python.org/dev/library/argparse.html}

\begin{verbatim}
#! /usr/bin/env python

#Header

__author__ = 'M.E. MySelf'
__version__ = 0.2

import numpy as np
import pylab as pl
import argparse

def mkplot(outfile,xx,yy1,yy2,yy3,yy4,ylab='Slope Unc. [e-/s]'):
    pl.loglog(xx, yy1, 'b--', xx, yy2, 'r:' , xx, yy3, 'g-', xx,  
        yy4, 'm-.', linewidth=3) 
    pl.legend(('Random Uncertainty','Correlated Uncertainty',  
        'Both','Equation'),'best')
    pl.ylabel(ylab) 
    pl.xlabel('Slope [e-/s]') 
    pl.savefig(outfile) 
    pl.clf()
    print 'Saved file to: ',outfile 
    return  
    
if __name__=='__main__': 

    parser = argparse.ArgumentParser(description='Make a plot.')
    parser.add_argument('-f','--file', default='Gordon2005_Fig16.txt', 
        type=str, help='Input file.')
    options = parser.parse_args()

    infile = options.file 
    slope_outfile = 'fig16_slope.pdf'  
    yint_outfile = 'fig16_yint.pdf'  
    
    slope, ran_slope_unc, corr_slope_unc, both_slope_unc,  \
        eqn_slope_unc, ran_yint_unc, corr_yint_unc,  \
        both_yint_unc, eqn_yint_unc = np.loadtxt(infile, 
        unpack=True) 
    mkplot(slope_outfile, slope, ran_slope_unc, corr_slope_unc,  
        both_slope_unc, eqn_slope_unc) 
    mkplot(yint_outfile, slope, ran_yint_unc, corr_yint_unc,  
        both_yint_unc, eqn_yint_unc,
        ylab='Y-Intercept Unc. [e-]')
\end{verbatim}

{\color{blue} {\sf\small EXERCISES}} \\
{\it Exercise \arabic{exercise} \stepcounter{exercise}:  \\
  Write your script MyFirstScript.py and execute it using all of the
  examples above.  See what happens if you enter something that is not
  allowed.  Edit your mkplot function so that it is more versatile.  }

\section{Error Handling and {\sf pdb.set\_trace()}}
Debugging can be a difficult and long process, but the {\sf\small pdb}
module can help.  {\sf\small pdb} is the Python debugger.  There are
many ways to use it, but a common method is with {\sf\small
  pdb.set\_trace()}.  To debug your code using this function, import
{\sf\small pdb} and insert the line \texttt{pdb.set\_trace()} into your
code before the part you are unsure about.  Execute the program.  Once
the interpreter reaches the {\sf\small pdb.set\_trace()} mark, it will
allow you to interact with the code and print variables as they are
defined.   Common {\sf\small pdb.set\_trace()} commands include:
\begin{itemize}
\item (n)ext:    continue to the next line
\item (l)ist:      list a few steps
\item (b)reak:  give line and file to break at
\item (s)tep:    moves into deeper calls (i.e. a function in NumPy, or
  our function {\sf\small mkplot}).
\item (c)ontinue: continue the program like normal.
\end{itemize}

{\color{blue} {\sf\small EXERCISES}} \\
{\it Exercise \arabic{exercise} \stepcounter{exercise}:  \\
  Add \texttt{import pdb} to your list of imports in your
  MyFirstScript.py file.  Insert the line \texttt{pdb.set\_trace()}.
  Execute your code and step through the program, print out variables
  to make sure they are what they are supposed to be, and step into
  your mkplot function.  Let the program continue. }


