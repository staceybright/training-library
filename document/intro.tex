\chapter{Basic Python Concepts}
\label{ch:intro}
\section{Introduction }
Python is a free, open-source, high level interpreted scripting language, similar to PERL. 
 However, Python is
also a high-level object-oriented programming language,
much like C++, Java, and Ruby.

The intention of this tutorial is to provide the Python basics
to understand how to perform simple astronomical data analysis.

We will discuss how to write and run Python scripts in Chapter~\ref{ch:scripts}.
Until then we will use python interactively, which is best for learning
python and for checking your code as you write it.

\subsection{A Few Notes on Python}
\begin{itemize}
\item Python is case sensitive.  This turns out not to be that
  difficult as Python is almost always written in lowercase.   
\item Python does not have curly brackets, or `BEGIN' and `END'
  statements.  It does however depend on indents.  Make sure to indent
  your code!
\item Comments in code are useful to help you remember what is going
  on, but also so that others know what you are doing.  In Python,
  comments begin with the \# symbol.
\item The line continuation symbol ` \textbackslash ' can be used to
  spread a command out over multiple lines.  This is not necessary
  inside parenthesis, brackets, or curly brackets as Python will wait
  for the closing symbol before assuming the end of the line.
\item Python indices start with zero.  
\end{itemize}

\section{Environment and Set Up}
\subsection{SSB Environment}
There are three different versions of the SSB (Science Software Branch) environments
available at the institute.  They are:
\begin{itemize}
\item SSBREL -- Updated as needed, standard publicly available release.
\item SSBX -- Updated weekly, a testing environment.  Holds latest
  bug fixes and updates.  Institute only release.
\item SSBDEV -- Updated daily.  Development updates intended for
  instrument teams to test bug fixes and upgrades.  Institute only
  release. 
\end{itemize}
Note that the version of Python related modules depend on the SSB
environment, especially if you are using NFS mounts.  To switch
to a particular version, type one of the following into your terminal: 
\begin{alltt}
\termtab ssbrel
\termtab ssbx
\termtab ssbdev
\end{alltt}

\subsection{Interactive Python Environment}
You will need to choose an interactive Python environment.  There are
two popular options.
\begin{itemize}
\item Python: \\
  To start the Python interactive environment, just type `python' at
  the prompt in your terminal.
\item iPython: \\
  To use the iPython interactive environment, type `ipython' instead.
  The advantage of iPython is that besides Python code, you are also
  able to execute UNIX commands such as `ls' and `cp'.  Furthermore,
  you can also log your session.  To start the log, type `\%logstart'
  in your iPython shell.  This will start logging into a default file
  named ipython.log.py.  The log-file follows a formatting convention
  of iPython and may not, on a first glance, look readable. However,
  iPython knows how to repeat the commands in your log-file when ran.
\end{itemize}
To exit either, type `exit()'.  Only once do we use the shell command
`cp', for which you would need to use iPython (or just do it in the
UNIX shell).  Otherwise either environment would work.

\section{Built-In Data Types}
\label{s:dtype}
The principal built-in types of Python are numerics, sequences,
mappings, files, classes, instances and exceptions.  Other data types
are available through importing modules (Example: the datetime
module).

There are four distinct numeric types in Python: plain integers, long
integers, floating point numbers, and complex numbers.  In addition,
Booleans are a subtype of plain integers.  Python fully supports mixed
arithmetic: when a binary arithmetic operator has operands of
different numeric types, the operand with the ``smaller'' type is
widened to that of the other.  Example:
\begin{alltt}
\pytab a = 2.0
\pytab b = 3
\pytab a * b
\end{alltt}

Any Python object can be tested for truth value with $in$, $if$ or
$while$ condition or as an operand of the Boolean operations.  Python
supports a vast number of truth values, but for simplicity I recommend
using $True$ and $False$.  General Boolean operations $or$, $and$, and
$not$ are also supported. Comparison operations are supported by all
objects.  They all have the same priority, which is higher than that
of the Boolean operations.  In Python, comparisons are done with
characters e.g. <, >=, ==, ! = (not like in IDL, where comparisons
are done with shortened keywords e.g. $eq$).

Examples of some of the most common data types are below.
\begin{itemize}
\item File: \\
  \texttt{\pytab infile = open('PythonTraining.pdf')} \\
  \texttt{\pytab infile.name} \\
  \texttt{\pytab infile.readline()}  \\
  \texttt{\pytab infile.close()}  
\item Integer: \\
 \texttt{\pytab a = 1} \\
 \texttt{\pytab a == 1}  \\
 \texttt{\pytab a.denominator}
\item Float: \\
  \texttt{\pytab a = 2.0}
\item String:  \\
  \texttt{\pytab a = 'three'} \\
  \texttt{\pytab infile = 'infile.fits'} \\
  \texttt{\pytab outfile = infile.replace('.fits', '\_DidWork.fits')} \\
  \texttt{\pytab 'Outfile: ', outfile} 
\item Tuple: An immutable list; elements can be of any data type. \\
  \texttt{\pytab a = (1, 2, 3)}  \\
  \texttt{\pytab a = (1, 2.0, 'three')}
\item List: A mutable list; elements can be of any data type. \\
  \texttt{\pytab a = [1, 2, 3]}  \\
  \texttt{\pytab b = [1, 2.0, 'three']}  \\
  \texttt{\pytab mylist = [] }  \\
  \texttt{\pytab mylist.append(3) } \\
  \texttt{\pytab print mylist } \\
  \texttt{\pytab mylist = [1, 2, 3, 4] } \\
  \texttt{\pytab len(mylist)} \\
  \texttt{\pytab yourlist = [5, 6, 7, 8, 9] } \\
  \texttt{\pytab mylist + yourlist }  \\
  \texttt{\pytab mylist.append(yourlist) }  \\
  \texttt{\pytab print mylist}  \\
  Notice that appending a list just adds one more element to `mylist',
  and this element is of type list (remember, an element in a list can
  be of any type).
\item Dictionary: A mapping which stores objects by a `key' and not
  position.  Below we show an example of a dictionary, and a
  dictionary of dictionaries.  We could also create a dictionary of
  lists.
\begin{alltt}
\pytab a = \{1:'one', 2:'two', 3:'three'\}
\pytab galaxy_dict = \{\} 
\pytab galaxy_dict 
\pytab galaxy_dict['M31'] = \{'RA':'00 42 44.33',
\ldots     'DEC':'+41 16 07.5','Vmag':3.44,'nickname':'Andromeda'\} 
\pytab galaxy_dict 
\pytab galaxy_dict['M104'] = \{'RA':'12 39 59.43185', 
\ldots    'DEC':'-11 37 22.9954','Vmag':3.44,'nickname':'Sombrero'\} 
\pytab galaxy_dict.keys()  
\pytab galaxy_dict['M31'].keys() 
\pytab galaxy_dict['M31']['Vmag']
\end{alltt}
\end{itemize}

{\color{blue} {\sf\small EXERCISES}} \\
{\it Exercise \arabic{exercise} \stepcounter{exercise}:  \\
Create a dictionary with strings as the keys.
}

\section{Objects, Instances, Attributes, Functions, and Methods, }
\label{s:attribute}

Python is an object oriented programming language and before we go any
further, it is important to note the differences and uses of objects,
instances, attributes, functions, and methods.
\begin{itemize}
\item Objects: \\
  Data carriers that also carry functions and attributes that work on
  that data.  In Section~\ref{s:dtype}, integers, floats, lists,
  tuples, and dictionaries are all objects.
\item Instance: \\
  In Section~\ref{s:dtype} when we said \texttt{galaxy\_dict = \{\}}
  we created an instance of a dictionary, $galaxy\_dict$, while
  \texttt{infile} is an instance of a $file$.
\item Attributes: \\
  {\sf \small denominator} is an attribute of an instance of a Python
  integer.  Therefore in Section~\ref{s:dtype} we used
  \texttt{a.denominator}.  Attributes do not have the `()' like
  functions do.  Also in Section~\ref{s:dtype} we used the file
  attribute {\sf\small name} when we said \texttt{infile.name}.
\item Functions: \\
  A function is a named piece of code that performs an operation and
  has open and closed parenthesis at the end (sometimes with variables
  inside).  {\sf \small len()} is a function, and in
  Section~\ref{s:dtype} we pass it the list, $mylist$.
\item Methods: \\
  A method is a function attached to an object that then has access to
  all other methods and attributes of that object.  Therefore we
  usually do not have to pass as much information to methods, and
  methods are easier to read.  All methods are functions, but not all
  functions are methods.  For example, in Section~\ref{s:dtype}, after
  we created the instance $mylist$, we append a number to that list
  using the method {\sf\small append()} with the line
  \texttt{mylist.append(3)}.  In Section~\ref{s:dtype} we also used
  the methods {\sf\small readline()} and {\sf\small close()} which
  belong to the file class.
\end{itemize}
Notice that we use dot notation to call an attribute or a method of a
particular instance.  As we will see in Section~\ref{s:mod}, dot
notation will also be used to call a method or a function from a
particular module.

\section{Double Underscore}
In Python you will notice certain names beginning and ending with
double underscores.  These names are used for attributes and methods
that are used or created by the interpreter (for a discussion on the
definition of attributes and methods see Section~\ref{s:attribute}).
Examples include:
\begin{itemize}
\item  \_\_file\_\_ :  an attribute automatically created by the interpreter
\item  \_\_add\_\_ : an attribute with special meaning to the interpreter
\item  \_\_init\_\_ : a method implicitly called by the interpreter,
  defined by the programmer
\end{itemize}
We will see examples of a few of these in Chapter~\ref{ch:scripts}.

\section{Common Built-In Functions and Statements}
For a complete list of Python built-in modules, see
\url{http://docs.python.org/library/index.html}, also listed in
Chapter~\ref{ch:links}.  Below we provide descriptions of some of the
most useful functions and statements.

%%%%%%%%%%%%%%%%%%%%%%%%%%%%%%%%%%%%%%%%  print
\subsection{ {\sf print } function}
{\color{blue} {\sf\small USE}} \\
The  {\sf\small print} function performs formatted output. 

{\color{blue} {\sf\small EXAMPLES}} 
\begin{alltt}
\pytab a = 3 
\pytab print a 
\pytab b = 2 
\pytab print a + b
\pytab print a / b
\pytab b = 2. 
\pytab print a / b
\pytab c = 'hi' 
\pytab print c + ' there' 
\pytab print 'I brought home \{\} flowers for \textbackslash 
\ldots    \$\{\} and all she said was \{\}.'.format(a,b+0.5,c)
\end{alltt}
This last statement is a bit more complicated.  The `\{\}' symbol says
to insert a variable from one of the arguments we give at the end.  In
the brackets we could specify data type, or which variable (0, 1, or
2), but we left it as default, which just takes the next argument in
line.  Notice the float does not print out exactly as we want.  We
would rather it say \$2.50.  Try this:

\texttt{\pytab print 'I brought home \{\} flowers for \$\{:.2f\} and all
  \textbackslash  \\
  \ldots    she said was \{\}.'.format(a, b+0.5, c)}
  
In this last example the `f' stands for `float', and the `.2'
says to include two decimal places.  For more information see: \\
\url{http://docs.python.org/library/string.html#formatstrings}.

{\color{blue} {\sf\small EXERCISES}} \\
{\it Exercise \arabic{exercise} \stepcounter{exercise}:  \\
Print out your own creative sentence.}

%%%%%%%%%%%%%%%%%%%%%%%%%%%%%%%%%%%%%%%%  help
\subsection{ {\sf help() } function}
{\color{blue} {\sf\small USE}} \\
The {\sf\small help()}  function gives the user information on many
aspects of the current Python session.  Type `q' to quit. 

{\color{blue} {\sf\small EXAMPLES}} 
\begin{alltt}
\pytab a = 3 
\pytab help(a) 
\end{alltt}

{\color{blue} {\sf\small EXERCISES}} \\
{\it Exercise \arabic{exercise} \stepcounter{exercise}:  \\
Type `help()' and explore the options.}

%%%%%%%%%%%%%%%%%%%%%%%%%%%%%%%%%%%%%%%%  range
\subsection{ {\sf range() } function}
{\color{blue} {\sf\small USE}} \\
The {\sf\small range()} function creates a list starting at zero of
the length you give it.  You may also pass the start, stop, and the
step you want to use (default of the step is 1).

{\color{blue} {\sf\small EXAMPLES}} 
\begin{alltt}
\pytab range(4) 
\pytab range(2, 8, 2) 
\end{alltt}
{\color{blue} {\sf\small SEE ALSO}} \\
Other useful similar functions in NumPy (Chapter~\ref{ch:numpy}) are
{\sf\small numpy.linspace}.

%%%%%%%%%%%%%%%%%%%%%%%%%%%%%%%%%%%%%%%%  dir
\subsection{ {\sf dir() } function}
{\color{blue} {\sf\small USE}} \\
The {\sf\small dir()} function returns the methods and attributes of
an object.
  
{\color{blue} {\sf\small EXAMPLES}} 
\begin{alltt}
\pytab a = range(10) 
\pytab dir(a) 
\end{alltt}

%%%%%%%%%%%%%%%%%%%%%%%%%%%%%%%%%%%%%%%%  len
\subsection{ {\sf len() } function}
{\color{blue} {\sf\small USE}} \\
The {\sf\small len()} function returns the number of elements
contained in an expression or variable.

{\color{blue} {\sf\small EXAMPLES}} 
\begin{alltt}
\pytab a = range(11) 
\pytab len(a) 
\end{alltt}
{\color{blue} {\sf\small SEE ALSO}} \\
Other useful similar functions in NumPy (Chapter~\ref{ch:numpy}) are
{\sf\small numpy.shape, numpy.size}.


%%%%%%%%%%%%%%%%%%%%%%%%%%%%%%%%%%%%%%%%  for loop
\subsection{ {\sf for } loop}
{\color{blue} {\sf\small USE}} \\
The {\sf\small for } statement is used to execute one or more
statements repeatedly, while incrementing or decrementing a variable
with each repetition, until a condition is met.

{\color{blue} {\sf\small EXAMPLES}} 
\begin{alltt}
\pytab a = range(11)
\pytab for x in a: print x
\pytab c = ['2005', '2006', '2007']
\pytab for x in c: print x + '-01'
\pytab b = [x*2 for x in a]  #Create a list in one line
\end{alltt}
  
{\color{blue} {\sf\small SEE ALSO}} \\
Other useful similar statements are {\sf\small if...elif...else, while}.  \\

{\color{blue} {\sf\small EXERCISES}} \\
{\it Exercise \arabic{exercise} \stepcounter{exercise}:  \\
Create a vector with 10 values equal to the square of the index.

Exercise \arabic{exercise} \stepcounter{exercise}:  \\
Create the following sequence using a {\sf\small for} statement:
2001-01-01, 2001-02-01, 2001-03-01, 2001-04-01.}



%%%%%%%%%%%%%%%%%%%%%%%%%%%%%%%%%%%%%%%%  xrange
\subsection{ {\sf xrange() } function}
{\color{blue} {\sf\small USE}} \\

The {\sf\small xrange()} function is similar to {\sf \small range()}
except that it does not create a new list (saving on memory) but
instead acts as a generator of numbers starting at zero up to the stop
value you pass it.  You may also pass the start, stop, and the step
you want to use (default of the step is 1).  This is used in iterations.

{\color{blue} {\sf\small EXAMPLES}} 
\begin{alltt}
\pytab for x in xrange(3): print x, x ** 2
\end{alltt}
Within a function, the {\sf\small for }  statement has the following
simple structure: 
\begin{alltt}
\pytab for x in xrange(1, 7): 
\ldots    a = x * 3. 
\ldots    print a 
\end{alltt}
  
{\color{blue} {\sf\small SEE ALSO}} \\
Other useful similar statements are {\sf\small range}.  \\

Notice that we use {\sf \small range()} when we use the actual list
created, but {\sf \small xrange()} when we just need an iterator.  

%%%%%%%%%%%%%%%%%%%%%%%%%%%%%%%%%%%%%%%%  try
\subsection{ {\sf try ... except } statement}
{\color{blue} {\sf\small USE}} \\
The {\sf\small try ... except} statement is used as an error
handler.  It will try whatever we put in the {\sf\small try} block,
but if whatever you assign as the `error' occurs, it will go to the
{\sf\small except} block.

{\color{blue} {\sf\small EXAMPLES}} \\
Try this code below.  Notice the error message you get.
\begin{alltt}
\pytab a = [1, 2, 3, 4, 5]  
\pytab for i in xrange(len(a)): 
\ldots    a[i + 1] = 100 + a[i] 
\ldots    print a[i], a[i + 1] 
\end{alltt}
To solve this problem, we put it in a {\sf\small try ... except}
statement: 
\begin{alltt}
\pytab a = [1, 2, 3, 4, 5] 
\pytab for i in xrange(len(a)): 
\ldots     try:  
\ldots         a[i + 1] = 100 + a[i] 
\ldots         print a[i], a[i + 1] 
\ldots     except IndexError: 
\ldots         print 'This index does not work: ',i + 1
\end{alltt}

{\color{blue} {\sf\small EXERCISES}} \\
{\it Exercise \arabic{exercise} \stepcounter{exercise}:  \\
  Imagine a situation where your code passes a function a variable,
  and that variable might be zero.  The only problem is that this
  function divides by that variable.
  
  Write a simple loop which calculates 1/$n$, where $n$ is what is
  passed.  What happens if $n=0$?  Write an error handling for this loop
  (hint: use {\sf\small try ... except} statement with
  ZeroDivisionError).}

\section{Importing Modules, and Common Functions}
\label{s:mod}
Python's built-in functions are limited.  The diversity of Python's
abilities come when we import modules.  The following syntax can be
used: 

\begin{alltt}
\pytab import <module>
\pytab from <module> import <function_in_module>
\pytab import <module> as <short_name>
\end{alltt}

%%%%%%%%%%%%%%%%%%%%%%%%%%%%%%%%%%%%%%%%  math.sqrt
\subsection{ {\sf math.sqrt() } function} 
{\color{blue} {\sf\small USE}} \\
If we want to take the square root of a number, we could
use the function {\sf\small sqrt()} in the module {\sf \small math}.  The following is two
examples of how to do this:
  
{\color{blue} {\sf\small EXAMPLES}} 
\begin{alltt}
\pytab import math
\pytab math.sqrt(100)
\end{alltt}
Notice we did not need the `.py' extension. If we do not need any other
math module, and we know we will not name a variable `sqrt' and
overwrite the function, we can do this:
\begin{alltt}
\pytab from math import sqrt
\pytab sqrt(100)
\end{alltt}

%%%%%%%%%%%%%%%%%%%%%%%%%%%%%%%%%%%%%%%%  glob.glob
\subsection{ {\sf glob.glob() } function}
{\color{blue} {\sf\small USE}} \\
The {\sf\small glob.glob()} function searches for files that match
the given path-name.  The path-name you give is a string similar to
the search strings used for the UNIX/Linux `ls'.

{\color{blue} {\sf\small EXAMPLES}} 
\begin{alltt}
\pytab import glob 
\pytab glob.glob('*') 
\pytab glob.glob('*.pdf')  
\pytab datadir = '/Users/username/data/'  #insert a usable path
\pytab glob.glob(datadir + 'MIR*12-03*ref??.fits')
\end{alltt}
{\color{blue} {\sf\small EXERCISES}} \\
{\it Exercise \arabic{exercise} \stepcounter{exercise}:  \\
Search for a set of files on your desktop.}

%%%%%%%%%%%%%%%%%%%%%%%%%%%%%%%%%%%%%%%%  random.random
\subsection{ {\sf random.random() } function}
{\color{blue} {\sf\small USE}} \\
The {\sf\small random.random()} function returns a random floating point
number in the range [0.0, 1.0). 
  
{\color{blue} {\sf\small EXAMPLES}} 
\begin{alltt}
\pytab import random 
\pytab random.random() 
\pytab print 'My random number between 2 and 8: ', \textbackslash
\ldots    2 + (8-2) * random.random()
\end{alltt}
{\color{blue} {\sf\small SEE ALSO}} \\
Other useful similar statements are {\sf\small random.uniform(), random.gaus()}.  \\

%%%%%%%%%%%%%%%%%%%%%%%%%%%%%%%%%%%%%%%%  re.search
\subsection{ {\sf re.search() } function}
{\color{blue} {\sf\small USE}} \\
The {\sf\small re} module is for `Regular Expression' operations.
It is used to work with strings, including sophisticated pattern
matching, as in the case of {\sf \small re.search()}.

{\color{blue} {\sf\small EXAMPLES}}  \\
\texttt{\pytab import re}  \\
\texttt{\pytab m = re.search('(?<=\_)$\textbackslash$d+','MIRI\_2011.fits')} \\
\texttt{\pytab m.group(0)}

In the above example we are looking for digits following an
underscore, `\_'.  The `\textbackslash d' is for digit, and the `+' is
for one or more.  The `(?<= \ldots )' matches if the position in the
string is proceeded by `\ldots'.
 
{\color{blue} {\sf\small SEE ALSO}} \\
Other useful similar statements are {\sf\small re.match()}. 

%%%%%%%%%%%%%%%%%%%%%%%%%%%%%%%%%%%%%%%%  os.ls
\subsection{ {\sf os.getcwd() and os.chdir() } functions}
{\color{blue} {\sf\small USE}} \\
The {\sf\small os} module is for `Operating System' operations.
Examples include {\sf \small os.getcwd()} which returns the current
working directory and {\sf \small os.chdir()} which changes the current
working directory.

{\color{blue} {\sf\small EXAMPLES}} 
\begin{alltt}
\pytab import os 
\pytab datadir = '/Users/username/data/'  #insert a usable path 
\pytab mydir = os.getcwd()  
\pytab if (mydir != datadir): os.chdir(datadir) 
\end{alltt}
{\color{blue} {\sf\small SEE ALSO}} \\
Other useful similar statements are {\sf\small os.open(),
  os.close()}. 

%%%%%%%%%%%%%%%%%%%%%%%%%%%%%%%%%%%%%%%%  sys.
\subsection{ {\sf sys.exit()} function}
{\color{blue} {\sf\small USE}} \\
The {\sf\small sys} module is for system-specific parameters and
functions.  It provides a way to interact with the interpreter.
{\sf\small sys.exit()} is different than {\sf\small exit()} in that it
will honor {\sf\small try} statements and you can
intercept the exit attempt.

{\color{blue} {\sf\small EXAMPLES}} 
\begin{alltt}
\pytab import sys
\pytab for i in xrange(10):
\ldots        if i <= 5:
\ldots            continue
\ldots        else:
\ldots            sys.exit('We do not need a number above 5.')
\end{alltt}




