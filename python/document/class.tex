\chapter{Classes}
\label{ch:classes}

\section{Class Definition}
\label{s:classes}
Classes are a very useful tool of Python.  They can mainly be seen as
a way to organize and simplify code.  You know you need to build a
class if:
\begin{itemize}
\item For a particular data type (i.e. FITS files, spectra, light
  curves, etc.) you are always calling the same functions.
\item You are having to pass these functions several variables.
\end{itemize}

The general format of a class, which I will call `SquareClass', and a
program that uses it, is given in the example below.  

\begin{verbatim}
class SquareClass(object):
    def __init__(self, s):  #s is a passed parameter
        self.side = s

    def GetArea(self):
        area = self.side * self.side
        return area

if __name__==`__main__':
    a = SquareClass(10)
    area = a.GetArea()
    print `The area of my square of side length: ', \
        a.side,'is: ',area
\end{verbatim}
{\color{blue} {\sf\small EXERCISES}} \\
{\it Exercise \arabic{exercise} \stepcounter{exercise}:  \\
  From the above code, give an example of an instance, an attribute,
  a method, and a function.}

Note that class definitions, like function definitions (def
statements), must be executed before they have any effect.  In
practice, the statements inside a class definition will usually be
function definitions.

In general, a class is initialized with a function {\sf \small
  \_\_init\_\_()}, which has at least one argument called $self$ (the
name self has absolutely no special meaning to Python, this is just a
convention).  When an {\sf \small \_\_init\_\_()} method has been
defined, class instantiation automatically invokes {\sf \small
  \_\_init\_\_()} for the newly-created class instance.

Notice that we do not have to pass the length of the side of the
square to the {\sf small GetArea} function.  This one of the benefits
of using classes.

\section{Class Inheritance}
Another ability of classes is class inheritance.  If you notice the
use of {\sf \small object} in {\sf \small SquareClass}, this is
actually the class inheriting from {\sf \small object}, which is the
default.  Class inheritance can also be used to extend existing
methods, or to overwrite functionalities of old classes.  For example,
say we have a new type of data, a cube for example, where the {\sf
  \small \_\_init\_\_()} method will work, but we will need a new {\sf
  \small GetArea()} method.  We can create a {\sf \small CubeClass}
which inherits from {\sf \small SquareClass} (which inherits from {\sf
  \small object}).

\begin{verbatim}
class SquareClass(object):
    def __init__(self, s):  #s is a passed parameter
        self.side = s

    def GetArea(self):
        area = self.side * self.side
        return area

class CubeClass(SquareClass):
    def GetArea(self): #Now surface area
        area = 6 * self.side * self.side
        return area

if __name__==`__main__':
    a = CubeClass(10)
    area = a.GetArea()
    print `The area of my cube of side length: ' \
        ,a.side,'is: ',area
\end{verbatim}

Notice that $CubeClass$ inherits $SquareClass$ just by passing the
class.  Since a new {\sf \small \_\_init\_\_()} method is not defined,
the old one is used.  We re-define {\sf \small GetArea()} which
overwrites the old one.

{\color{blue} {\sf\small EXERCISES}} \\
{\it Exercise \arabic{exercise} \stepcounter{exercise}:  \\
  An excellent example of when you might want to write a class is for
  FITS files.  Create a FitsClass.  For the {\sf \small
    \_\_init\_\_()} method, pass the filename.  Initialize your header
  and data (self.header and self.data).  Create another method in that
  class which will save a FITS file to a new filename, which you pass
  it.}


