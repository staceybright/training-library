\chapter{PyRAF}
\label{ch:pyraf}
 
\section{PyRAF Introduction}

PyRAF calls IRAF tasks from Python.  One of the main motivations for
creating PyRAF is to create a version of IRAF that is compatible with
a programming language other than CL. CL has many short comings which
are more and more apparent in more complicated programs.

PyRAF has its own interactive PyRAF session.  To start it, just type
`pyraf' in a terminal.  This should look familiar to you if you know
IRAF.  Type `.exit' to exit.  The remaining part of this chapter will
have you import PyRAF into Python unless otherwise stated.

\section{A PyRAF Example with {\sf iraf.daofind()}}

Again, create a file in your script directory, and open it in your
favorite editor.  Add your header.  For your 'ABOUT' section, we will
be creating a script showing how python can be used to execute
IRAF/PyRAF tasks. We will use some JWST/MIRI data, n9vf01hfq\_ima.fits,
which can be found in /user/gunning/Python\_Training/.

Import the PyRAF module and from within that module import the IRAF
instance.  The code for importing these functions looks like this:

\begin{verbatim}
import pyraf 
from pyraf import iraf
from iraf import noao, digiphot, daophot
\end{verbatim}

As you can infer `iraf' is something that lives inside of `pyraf'.
For example, if you want to run {\sf\small phot} on some files you
need to create a list of coordinates.  To do this we will use
{\sf\small daofind}.  Since {\sf\small daofind} is an iraf module, we
execute it using dot notation, like this:

\texttt{iraf.daofind(parameter1,parameter2,parameter3)}

The parameters for a PyRAF task executed in the python environment are
the same as they would be in the PyRAF environment. Pull up a PyRAF
window and open the help file to see what the parameters for
{\sf\small iraf.daofind()} are. Now are going to run {\sf\small
  iraf.daofind()} on the file n9vf01hfq\_ima.fits.  

\begin{verbatim}
iraf.daofind(image='n9vf01hfq_ima.fits[1]')  
\end{verbatim}

The `[1]' is for extension `1'.  You should be prompted for several
parameters, just hit `enter' to go through the defaults.  Type `ls' in
the directory where you ran this -- there should now be a file called
n9vf01hfq\_ima.fits1.coo.1. In PyRAF open up the file and look at the
output, then compare this with the image on DS9.  Did {\sf\small
  iraf.daofind()} do a good job?  You can facilitate this process with
the {\sf\small tvmark} task.

\begin{alltt}
\termtab pyraf  
\pyraftab images 
\pyraftab !ds9 & 
\pyraftab display n9vf01hfq_ima.fits[1] 1
\end{alltt}

The last '1' is for which DS9 frame to display in.

\begin{alltt}
\pyraftab tv  
\pyraftab tvmark 1 n9vf01hfq_ima.fits1.coo.1
\end{alltt}

We can see that we did not to too well, but let's ignore that for now
and work on our program.

Go ahead and run this program again, and you will see that it will
create a second file called n9vf01hfq\_ima.fits1.coo.2. What if we did
not want so many files and instead wanted to always write to
n9vf01hfq\_ima.fits1.coo.1?  IRAF might have a problem overwriting
this file, so a simple solution would be to remove it.  For this we
will need to import {\sf\small os}.  Furthermore, if we don't want to
have to press `enter' at the prompt, we can add to our program the
extra parameters it is looking for.  Finally, let's write this program
so that we can run the process on multiple files.  For that we will
import {\sf\small glob}.

Here is what we have:

\begin{verbatim}
#! /usr/bin/env python
# HEADER

#Load the packages we need 
import pyraf, os, glob  
from pyraf import iraf
from iraf import noao, digiphot
  
#Generate a list of all the fits files
file_list = glob.glob('*_ima.fits') 
print file_list 

#Loops through all the .fits files 
for ima in file_list: 
    #Test for old files, and remove them if they exist
    file_query = os.access(ima + '1.coo.1', os.R_OK)  
    if file_query == True: 
        os.remove(ima + '1.coo.1')  
    #Run daofind on one image  
    iraf.daofind( 
        image = ima + '[1]',  
        interactive = 'no',  
        verify = 'no')  
\end{verbatim}

{\color{blue} {\sf\small EXERCISES}} \\
{\it Exercise \arabic{exercise} \stepcounter{exercise}:  \\
  Write your script that uses {\sf\small iraf.daofind()}.}

Anytime you change the default settings to a PyRAF command it is a
good idea to change them back.  You can do this with the {\sf\small
  iraf.unlearn()} command as shown below.

\begin{alltt}
\pyraftab iraf.daofind.unlearn()
\pyraftab iraf.unlearn('daofind')
\end{alltt}






