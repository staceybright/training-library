\chapter{Handeling FITS files and ASCII data tables using Astropy}
\label{ch:fits}
Astropy is a python library for astronomy developed by professional astronomers
and software developers from around the world, some of which work here at STScI
in the Science Software Branch.  It is under continuous development and is quickly
becoming a powerful library, especially for handeling FITS files and ASCII tables.
Visit the website listed in Chapter~\ref{ch:links} for more information and useful
documentation.

The astropy.io.fits module provides an interface to FITS formatted files under the 
Python scripting language.  astropy.io.fits data structures are a subclass of NumPy 
arrays, which means that they can use NumPy arrays' methods and attributes.  The
astropy.io.ascii module provides flexible and easy-to-use methods for reading and 
writing ASCII data tables.  In the following sections, we will explore these two modules.
 
\section{Opening, Reading, and Closing a FITS File}
As an example, we will use data from the \emph{NICMOS} instrument located here:  \\
/grp/jwst/wit/miri/randers/PythonTraining/n9vf01hfq\_ima.fits \\ 
Please copy this file to your working directory. 

Below we show an example of opening a FITS file, getting
the data and the header, closing the file, printing out the shape of
the data using {\sf \small numpy.shape}, printing out header values,
and finally making changes to the data.

\begin{alltt}
\pytab from astropy.io import fits
\pytab infile = 'n9vf01hfq_ima.fits'
\pytab fits.info(infile)
\pytab data = fits.getdata(infile, 1) 
\pytab hdr = fits.getheader(infile, 0) 
\pytab data.shape
\pytab print hdr 
\pytab hdr['DARKCORR'] 
\pytab hdr['DARKCORR'] = 'PERFORM'
\pytab hdr['DARKCORR']
\pytab print data[-2:]  #print the last two lines.
\pytab data[-1:][0][0] = 0
\pytab print data[-1]
\end{alltt}

Notice that $hdr$ behaves like a dictionary.  We did some
bad things to this file, but let's save it anyway to a new file.

\begin{alltt}
\pytab outfile = 'mybad.fits'
\pytab fits.writeto(outfile, data, hdr)
\pytab print 'Saved FITS file to: {}'.format(outfile)
\end{alltt}

Alternatively, if we want to modify the original file directly,
we can do the following:

\texttt{\pytab fits.update(infile, data, 1)}

\section{{\sf fits.getval()} and {\sf fits.setval()} functions}

If you are familiar with IRAF, you are probably familiar with IRAF's
{\sf\small hedit} function, which allows you to add, delete, and
modify keywords in a FITS header.  

First, lets take a look of our example file's header using {\sf\small
  imheader} in PyRAF.  In PyRAF navigate to the folder where your
n9vf01hfq\_ima.fits file is located, and try the following:

\begin{alltt}
--> imheader n9vf01hfq_ima.fits[0] l+ | page 
\end{alltt}

We see that there is a 'NSLEWCON' keyword, and it is set to 'Clear.'
Using {\sf\small hedit} we can change the 'NSLEWCON' keyword from
'Clear' to 'Set' as shown here:

\begin{alltt}
--> hedit n9vf01hfq_ima.fits[0] NSLEWCON 'Set'  \textbackslash
\ldots     verify=no update=yes
\end{alltt}

In the above example we made sure the 'update' parameter was set to
'yes.'  We can check that our edit was successful by using  {\sf\small
  imheader} again:

\begin{alltt}
--> imheader n9vf01hfq_ima.fits[0] l+ | page 
\end{alltt}

Without using IRAF, there is a simple way to do this in astropy using
{\sf\small fits.getval()} and {\sf\small fits.setval()}, shown in
the example below.

\begin{alltt}
\pytab from astropy.io import fits
\pytab infile = 'n9vf01hfq_ima.fits' 
\pytab key = 'NSLEWCON'
\pytab fits.getval(infile, key, 0)
\pytab fits.setval(infile, key, value='Clear', ext=0)
\pytab fits.getval(infile, key, 0)
\end{alltt}

Now our FITS file is back to its 'initial' state.  No harm done.


\section{Reading and Writing ASCII Data Files}

The astropy.io.ascii module provides two robust methods, ascii.read() and
ascii.write(), for reading and writing multi-column delimited data tables,
respectively.