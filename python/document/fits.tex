\chapter{Handeling FITS files and ASCII data tables using Astropy}
\label{ch:fits}
Astropy is a python library for astronomy developed by professional astronomers
and software developers from around the world, some of which work here at STScI
in the Science Software Branch.  It is under continuous development and is quickly
becoming a powerful library, especially for handeling FITS files and ASCII tables.
Visit the website listed in Chapter~\ref{ch:links} for more information and useful
documentation.

The {\sf \small astropy.io.fits} module provides an interface to FITS formatted files under the 
Python scripting language.  {\sf \small astropy.io.fits} data structures are a subclass of NumPy 
arrays, which means that they can use NumPy arrays' methods and attributes.  The
{\sf \small astropy.io.ascii} module provides flexible and easy-to-use methods for reading and 
writing ASCII data tables.  In the following sections, we will explore these two modules.
 
\section{Opening, Reading, and Closing a FITS File}
As an example, we will use data from the \emph{WFC3} instrument located here:

/user/gunning/Python\_Training/icft01crq\_flt.fits

Please copy this file to your working directory. 

Below we show an example of opening a FITS file, getting
the data and the header, closing the file, printing out the shape of
the data using {\sf \small numpy.shape}, printing out header values,
and finally making changes to the data.

\begin{alltt}
\pytab from astropy.io import fits
\pytab infile = 'icft01crq_raw.fits'
\pytab fits.info(infile)
\pytab hdulist = fits.open(infile)
\pytab hdr = hdulist[0].header  # Get the primary header
\pytab data = hdulist[1].data  # Get the data from the 1st extension 
\pytab data.shape
\pytab print hdr 
\pytab hdr['FLSHCORR'] 
\pytab hdr['FLSHCORR'] = 'PERFORM'
\pytab hdr['FLSHCORR']
\pytab print data[-2:]  # Print the last two lines.
\pytab data[-1:][0][0] = 0
\pytab print data[-1]
\end{alltt}

Notice that $hdr$ behaves like a dictionary.  We did some
bad things to this file, but let's save it anyway to a new file.

\begin{alltt}
\pytab outfile = 'mybad.fits'
\pytab fits.writeto(outfile, data, hdr)
\pytab print 'Saved FITS file to: {}'.format(outfile)
\end{alltt}

Alternatively, if we want to modify the original file directly,
we can do the following:

\texttt{\pytab fits.update(infile, data, 1)}

As a word of caution, note that {\sf \small astropy.io.fits} reads in FITS images
as {\sf \small(rows, cols)} or {\sf \small (y, x)}, not {\sf \small (x, y)}.  
This is often a `gotcha' for users who are indexing spacific areas of the array.

\section{{\sf fits.getval()} and {\sf fits.setval()} functions}

If you are familiar with IRAF/PyRAF, you are probably familiar with IRAF's
{\sf\small hedit} function, which allows you to add, delete, and
modify keywords in a FITS header.  

First, lets take a look of our example file's header using {\sf\small
  imheader} in PyRAF.  In PyRAF navigate to the folder where your
icft01crq\_raw.fits file is located, and try the following:

\begin{alltt}
--> imheader icft01crq_raw.fits[0] l+ | page 
\end{alltt}

We see that there is a `DARKFILE' keyword, and it is set to `iref\$y2j13512i\_drk.fits.'
Say we wanted to recalibrate this file using a different `DARKFILE'.  The value of this
keyword can be changed using {\sf\small hedit}, as shown here:

\begin{alltt}
--> hedit icft01crq_raw.fits[0] DARKFILE 'iref$y2p1831ci_drk.fits'  \textbackslash
\ldots     verify=no update=yes
\end{alltt}

In the above example we made sure the `update' parameter was set to
`yes.'  We can check that our edit was successful by using  {\sf\small
  imheader} again and checking the value of the `DARKFILE' keyword:

\begin{alltt}
--> imheader icft01crq_raw.fits[0] l+ | page 
\end{alltt}

Using Python, there is a simple way to do this with {\sf \small astropy.io.fits} 
using {\sf \small fits.getval()} and {\sf\small fits.setval()}, shown in
the example below.

\begin{alltt}
\pytab from astropy.io import fits
\pytab infile = 'icft01crq_raw.fits[0]' 
\pytab key = 'DARKFILE'
\pytab fits.getval(infile, key, 0)
\pytab fits.setval(infile, key, value='iref$y2j13512i_drk.fits', ext=0)
\pytab fits.getval(infile, key, 0)
\end{alltt}

Now our FITS file is back to its `initial' state.  No harm done.

\section{Reading and Writing ASCII Data Files}

The {\sf \small astropy.io.ascii} module provides two robust methods, 
{\sf \small ascii.read()} and {\sf \small ascii.write()}, for reading and 
writing multi-column delimited data tables, respectively.

%%%%%%%%%%%%%%%%%%%%%%%%%%%%%%%%%%%%%%%%  ascii.read
\subsection{ {\sf astropy.io.ascii.read()} function}
{\color{blue} {\sf\small USE}} \\
The {\sf\small astropy.io.ascii.read()} function reads in a table of data 
from a specified file and returns an astropy.Table object.

As an example, we will use these files:

/user/gunning/Python\_Training/flux\_vs\_time\_A.dat \newline
/user/gunning/Python\_Training/flux\_vs\_time\_C.dat

These data are used to plot the flux versus time for a standard star and serves as a
monitor of the WFC3/UVIS photometric stability for amps A and C, respectively.  In 
this section, we will read in the data, and in chapter 4, we will use it to produce 
the plots.
  
{\color{blue} {\sf\small EXAMPLES}} 
\begin{alltt}
\pytab from astropy.io import ascii
\pytab infile = 'flux_vs_time_A.dat'
\pytab data = ascii.read('flux_vs_time_A.dat', 
                names=['MJD', 'Flux_diff', 'Flux_err', 'Flux_linear_fit'])
\pytab print data
\pytab print data['MJD']
\pytab print data['MJD', 'Flux_diff']
\pytab print data['Flux_diff'] * 10
\pytab pos_flux = data['Flux_diff'] > 0
\pytab print data[pos_flux]
\end{alltt}

{\color{blue} {\sf\small EXERCISES}} \\
{\it Exercise \arabic{exercise} \stepcounter{exercise}:  \\
Try reading in the data from `flux\_vs\_time\_C.dat' using {\sf \small ascii.read()}}

%%%%%%%%%%%%%%%%%%%%%%%%%%%%%%%%%%%%%%%%  ascii.write
\subsection{ {\sf astropy.io.ascii.write() } function}
{\color{blue} {\sf\small USE}} \\
Similar to {\sf\small ascii.read()}, the {\sf\small astropy.io.ascii.write()} function writes a table of data 
to a specified file.  For fun, lets try taking the MJDs from `flux\_vs\_time\_A.dat', 
normalizing them to the first observation date, and writing the new table to a new text file:

{\color{blue} {\sf\small EXAMPLES}} 
\begin{alltt}
\pytab from astropy.io import ascii
\pytab infile = 'flux_vs_time_A.dat'
\pytab data = ascii.read('flux_vs_time_A.dat', 
              names=['MJD', 'Flux_diff', 'Flux_err', 'Flux_linear_fit'])
\pytab first_date = min(data['MJD'])
\pytab data['MJD'] = data['MJD'] - first_date
\pytab print data
\pytab ascii.write(data, 'flux_vs_time_A_mjdnorm.dat')
\end{alltt}

{\color{blue} {\sf\small EXERCISES}} \\
{\it Exercise \arabic{exercise} \stepcounter{exercise}:  \\
Try making a MJD-normalized text file for the `flux\_vs\_time\_C.dat' data using {\sf \small ascii.write()}}