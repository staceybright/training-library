\chapter{Introduction to PyRAF}
\label{cha:chapter2}


 
\section{What is PyRAF? }
PyRAF is a command language for running IRAF tasks in a Python like environment. It works very similar to IRAF, but has been updated to allow such things as importing Python modules, starting in any directory, GUI parameter editing and help. Most importantly, it can be imported into Python allowing you to run IRAF commands from within a larger script. PyRAF is part of the stsci\_python package and is a product of the Science Software Branch right here at STScI. All of the STScI calibration pipelines can be run from PyRAF although they are now written in Python and more recent pipelines (e.g. \emph{Calcos}) can be run as stand-alone programs.

\subsection{IRAF}
IRAF stands for Image Reduction and Analysis Facility and is written and supported by the National Optical Astronomy Observatories (NOAO). It has many tools for both spectroscopic and image analysis. Scripting in IRAF can be done using the CL scripting, however that will not be covered in this document. Most information covered in this document applies to both IRAF and PyRAF but is generally more user friendly in PyRAF.

\section{Getting Started}
\subsection{login.cl file}
All of your PyRAF preferences are held in a file with the name login.cl.  If you are running IRAF, then your login.cl file must be in the directory in which you start IRAF. PyRAF can be called from any directory as long as your login.cl file is located in an directory called iraf in your home directory. When you start PyRAF it will first look for a login.cl file in the directory in which you started it (this allows you to have different configurations for different projects). If it doesn't find a login.cl file there it will then look in your /home/iraf directory (where home is your login name). You can use your login.cl file to set all kinds of preferences (look inside it, most things are pretty self explanatory, but ask if you have questions). In here you can also add packages and set paths to commonly used directories (such as oref) you can see some are already set up.

\subsection{SSBREL, SSBX, and SSBDEV}
There are three environment which you can use which get updated at various times. These set the versions of STSDAS, STSCI\_PYTHON, PyRAF, and PyFITS. The SSBREL environment contains packages which have been released to the public. This generally happens a few times a year. This is the most stable (but least up to date) environment. The SSBX environment is updated weekly and may contain some bugs but is generally pretty stable. This is the environment chosen by most people at the institute. The SSBDEV environment is updated daily. More details can be found at: \url{http://ssb.stsci.edu/ssb_software.shtml}

Installed copies of SSBREL, SSBX, and SSBDEV are not automatically updated.  To get an up-to-date version of a particular environment, simply use `ur\_update` in the environment you wish to update.  To switch between environments, use `ur\_setup common <environment>'

When you start up PyRAF, the environment will be printed prominently in the startup text.

\subsection{Exercises}
\begin{enumerate}
\item Find and open your login.cl file.  See if you can find which editor is used by default. Are any packages loaded automatically? 
\item Try switching between SSBREL, SSBX, and SSBDEV. Open PyRAF in each to confirm that you switched.
\end{enumerate}

\section{The Basics}
\subsection{Navigation}
PyRAF contains many packages which have to be imported. You import a package by typing the name in the command line. For example, to use the {\bf Calstis} function I need to have the {\bf stis} packge loaded which is in the {\bf hst\_calib} package, which is in the {\bf stsdas} package,  I type the following in a PyRAF window:

\begin{minipage}{4in}
\setlength{\oddsidemargin}{0.25 in}
\setlength{\evensidemargin}{0.25 in}
\begin{tabular}{ll}
& {\color{RoyalBlue}--> stsdas}\\
& {\color{RoyalBlue}--> hst\_calib}\\
&{\color{RoyalBlue}--> stis}\\
\end{tabular}
\end{minipage}

To find out the packages available in your current package type \emph{?}. For example:

\begin{minipage}{4in}
\setlength{\oddsidemargin}{0.25 in}
\setlength{\evensidemargin}{0.25 in}
\begin{tabular}{ll}
& {\color{RoyalBlue}--> stsdas}\\
& {\color{RoyalBlue}--> hst\_calib}\\
& {\color{RoyalBlue}--> ?}\\
\end{tabular}
\end{minipage}

This will tell you all of the packages in {\bf hst\_calib}. Functions with an @ symbol are parameter tables, not functions.

\subsection{Help}
You get help information on any package or function by typing {\bf help function}. This not only tells you about a function, it also tells you the packages that you have to call to get to it. For instance, if you want to use the command {\bf splot}:

\begin{minipage}{4in}
\setlength{\oddsidemargin}{0.25 in}
\setlength{\evensidemargin}{0.25 in}
\begin{tabular}{ll}
& {\color{RoyalBlue}--> help splot}
\end{tabular}
\end{minipage}

This will tell you that {\bf splot} is in the {\bf onedspec} in the {\bf noao} package. This can also be entered in GUI form from the epar GUI. 

\subsection{epar}
All functions have input parameters most of which can either be entered in the command line or can be set in the parameter editor using the epar GUI. If you type: \emph{epar function} you will get a table of all of the parameters. There are 5 buttons at the top of the epar editor: Execute, Save, Unlearn, Cancel, and Help. Execute runs the function with the parameters given, Save exits the epar editor and saves the parameters but doesn't execute the function, Unlearn set all of the parameters to their default values, Cancel exits the epar editor without saving, and Help displays the help file for the function. 

\subsection{Using a List as Input}
Many PyRAF functions cannot use wildcards but can take a file which is a list of files. A file which is a list is distinguished from an input file for processing by using an @ sign at the beginning of the file name. For example if you have 3 files, file1.fits, file2.fits, and file3.fits, you can first make a file which is a list of the file names:

\begin{minipage}{4in}
\setlength{\oddsidemargin}{0.25 in}
\setlength{\evensidemargin}{0.25 in}
\begin{tabular}{ll}
& {\color{RoyalBlue}> ls *.fits > filelist.txt } \\
\end{tabular}


Then use this as the input to the {\bf catfits} function: \\
\begin{tabular}{ll}
& {\color{RoyalBlue}--> catfits @filelist.txt } \\
\end{tabular}
\end{minipage}


\subsection{importing function}
Because PyRAF is Python based, you can import any Python module. This is particularly useful when calibrating a group of datasets as many instrument pipelines (Cal*) will not take wild cards as inputs.  If you are mixing PyRAF and Python commands you often have to put an iraf. infront of the commands to let PyRAF know that your are executing a PyRAF command. In the following example I will import the {\bf glob} module from python and use it to select all of the raw files in a directory. I will then calibrate each raw file (note I use iraf.calstis rather than just calstis).

\begin{minipage}{4in}
\setlength{\oddsidemargin}{0.25 in}
\setlength{\evensidemargin}{0.25 in}
\begin{tabular}{ll}
& {\color{RoyalBlue}> pyraf}\\
& {\color{RoyalBlue}--> import glob}\\
& {\color{RoyalBlue}--> flist = glob.glob('*raw.fits')}\\
& {\color{RoyalBlue}--> for ifile in flist:}\\
\end{tabular}
\setlength{\parindent}{0.5 in}
%\setlength{\parskip}{-0.2 in}

{\color{RoyalBlue}iraf.calstis(ifile)}\\
\end{minipage}

\subsection{Exercises}
\begin{enumerate}
\item In PyRAF located the package {\bf apextract}. What tasks are in this package? What does the {\bf aptrace} task do? Are there any parameter tasks?
\end{enumerate}

