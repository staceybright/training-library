\chapter{IPython Notebook}
\label{ch:notebook}

\section{Getting Started}

IPython Notebook is an interactive web-based tool which combines code execution, mathematics, and rich media output.

You can find a plethora of information at \url{http://ipython.org/ipython-doc/dev/notebook/index.html}

Let's get started.  The data needed for this chapter can be found in /user/gunning/Python\_Training/.

To open a notebook window, open a terminal and initiate Ureka:

\begin{alltt}
\termtab ur_setup common ssbx
\termtab ipython notebook
\end{alltt}

Opening the notebook this way will create a new tab in your browser. All subsequent interactions will now
be through the browser and the terminal will be running the server that the browser interacts with.

IPython Notebook will look for notebook files (.ipynb) in the directory where you started the notebook in.
Once in the notebook window, to execute a cell you must hit shift+enter.

\section{Plotting Inline}

A unique feature that the notebook has is the ability to plot inline.

You can initiate inline plotting two ways.

From the terminal:
\begin{alltt}
\termtab ur_setup common ssbx
\termtab ipython notebook --pylab inline
\end{alltt}

From the notebook window:
\begin{alltt}
\pytab \%pylab inline
\end{alltt}

Without telling the notebook to plot inline, plots will pop up in a separate window just as they normally
would plotting in the python terminal.

\section{Notebook Examples}

\begin{figure}[H]
  \centering
    \includegraphics[width=0.98\textwidth]{notebook1.png}
\end{figure}

\newpage
\begin{figure}[H]
  \centering
    \includegraphics[width=0.95\textwidth]{notebook2.png}
\end{figure}

\begin{figure}[H]
  \centering
    \includegraphics[width=0.95\textwidth]{notebook4.png}
\end{figure}

\begin{figure}[H]
  \centering
    \includegraphics[width=0.95\textwidth]{notebook5.png}
\end{figure}

\newpage
{\color{blue} {\sf\small EXERCISES}} \\

{\it Exercise \arabic{exercise} \stepcounter{exercise}:  \\
Change the colormap for the figure in line 7 to whatever map you prefer. 
(a quick google of 'matplotlib colormap' will give you the list of available maps)}
\\

{\it Exercise \arabic{exercise} \stepcounter{exercise}:  \\
Use the same colormap for the figure in line 8. Add '\_r' to the end of your colormap name.
What did this do to your image? - Write answer in your notebook}
\\
\\
{\it Exercise \arabic{exercise} \stepcounter{exercise}:  \\
Pick  different rows to plot for the figure in line 18 and change the line colors.
\\
Save your notebook (lastname\_training.ipynb) and place it in /user/gunning/Python\_Training/notebooks/}

\let\cleardoublepage\clearpage

%%%%%%%%%%%%%%%%


